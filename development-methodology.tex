\section{TDD}

In questa sezione descriver\`o la metodologia di sviluppo che ho
adottato per sviluppare la fase di implementazione.

\subsection{Domain modelling and big-design-up-front? No thanks!}
Ho avuto delle esperienze lavorative e in alcuni corsi affrontati
durante il mio percorso di studio, dove si prediligeva un approccio
pi\`u "classico" all'ingegnerizzazione del software. Questo approccio
era composto da una consistente fase di analisi prima di avere
feedback dallo strumento pi\`u veritiero, il codice. 

Durante questa fase di analisi si identificavano le varie entit\`a che
nello sviluppo del progetto si sarebbero tramutate in classi, si
attribuivano le responsabilit\`a in modo tale che il codice dovesse
adattarsi a quanto dettato dall'analisi (\emph{e non il
  contrario...}).

Questo approccio ritengo che sia molto rigido e che non permetta uno
sviluppo dinamico e \emph{istruttivo} come invece si mira ad avere con
tecniche \emph{agile}. Per questo motivo quello che ho cercato di
seguire \`e stata la metodologia \emph{Test-Driven Development}, che
permette un "dialogo" stretto tra le idee che si vogliono implementare
e il codice che effettivamente abbiamo scritto, avendo da quest ultimo
molti feedback che in molte situazioni hanno aiutato a chiarificare le
idee di partenza.

Ho sempre avuto molta curiosit\`a sui metodi \emph{agile} e,
soprattutto, sulla metodo \emph{Test Driven Development}. Non ho mai
avuto opportunit\`a di affrontare un progetto usando queste linee idee
e ho pensato che questo lavoro potesse essere un buon banco di prova.

Il volume su cui ho cercato di apprendere questa tecnica \`e
\footnote{aggiungere riferimento bibliografico qui al volume scritto
  per intero} \emph{Test Driven Development: By Example} scritto da
Kent Beck. In questo scritto vi sono molti esempi, scritti usando il
linguaggio di programmazione \emph{Java}, anche se le idee di avere
test automatici per assicurare la qualit\`a del codice erano gi\`a
state battute in \emph{Lisp} e, dallo stesso Beck, in
\emph{Smalltalk}.

Riporto la prima frase della prefazione:
\begin{quotation}
  \emph{Clean code that works}, in Ron Jeffries' pithy phrase, is the
  goal of Test-Driven Development. Clean code that works is a
  worthwhile goal for a whole bunch of reasons...
\end{quotation}
Non posso certo dire che le mie implementazioni soddisfino la
citazione precedente (almeno per la parte \emph{that works}), per\`o
aver guidato l'intero sviluppo tramite test \`e stata una esperienza
molto istruttiva, di cui sono soddisfatto.

Nelle prossime sezioni descrivo come i \emph{tests} sono stati di
grande aiuto durante il lavoro.

\subsection{Learning tests}
Questo tipo di test sono stati scritti soprattutto durante i primi
momenti di sviluppo, e hanno avuto l'obiettivo di
familiarizzare l'utilizzo della libreria \emph{JSBML}. 

La documentazione fornita \`e estesa e ben dettagliata, per\`o ci
sono stati degli aspetti che senza aver esercitato alcuni test non era
possibile studiarli a priori.

Per questi tests ho dedicato un package dedicato, in modo da
fattorizzare in un unico contenitore tutta la logica che ho avuto
bisogno per chiarire tutti gli aspetti necessari all'avanzamento del
mio lavoro.

In \footnote{aggiungere qui riferimento bibliografico a TDD, pag 136}
viene spiegato questo "pattern" e un aspetto che non ho potuto
sperimentare direttamente in questo sviluppo, ma che penso sia
interessante notare. Se viene rilasciata una nuova versione della
libreria \emph{JSBML} abbiamo il grande vantaggio di avere delle
asserzioni che verificano i concetti che abbiamo utilizzato come
dipendenze per il nostro lavoro. Siamo interessati che nessuna di
queste \emph{assert} sia violata dal nuovo rilascio. 

Senza questi test avremo dovuto provare il software manualmente per
controllare che il comportamento non sia cambiato. Avendoli invece,
\`e sufficiente far "correre" di nuovo tutta la batteria per
assicurare che i concetti su cui ci siamo basati non abbiano subito
modifiche.

\subsection{Pros}
Il maggior vantaggio che ho avuto utilizzando questa metodologia \`e
stato di poter pensare ad una funzionalit\`a vestendo due
"cappelli". Dapprima ho potuto concentrarmi sul comportamento che
desideravo, quindi poter pensare solo il sistema di messaggi che i
miei oggetti si sarebbero scambiati (e questo corrisponde alla fase
\emph{test-first}), creando un \emph{test method} per catturare
l'idea.

Quando vedevo il nuovo test fallire, ho potuto dedicarmi
all'implementazione necessaria per passare il test. 

A differenza di quanto suggerisce Beck, non ho quasi mai applicato il
pattern \emph{Fake it, ('Til you make it)} \footnote{aggiungere qui
  riferimento a TDD, pag. 151}, in modo da arrivare subito alla
\emph{green bar} e procedere poi alla rimozione dei eventuali
duplicazioni o \emph{code smell} mediante refactoring, perch\`e in
molti casi l'implementazione necessaria per superare il test non era
cosi complessa da dover procedere in \emph{baby step} (usando la
terminologia di Beck).

Forse sono stato un po' troppo prolisso sotto alcuni aspetti: le mie
batterie di test contano un totale di 191 metodi di test. Magari si
sarebbe potuto risparmiare qualcosina, per\`o quasta granularit\`a
fine penso sia stata di aiuto nella fase di superamento del test, la
maggior parte delle volte la logica da implementare non \`e stata
difficoltosa.

\subsection{Cons}
Una nota negativa che ho registrato riguarda il grado di profondit\`a
(se si vuole di precisione) che ho associato ai test pi\`u critici.

Nella parti pi\`u delicate mi sono sempre chiesto se i test scritti
era sufficienti a catturare il comportamento che effettivamente avrei
voluto. In alcuni punti ho dovuto ricorrere al debugger, ma penso che
questo sia accettabile, per aiutarmi a capire se qualche aspetto
necessit\`a di maggior copertura da parte dei metodi di test.

Parte di questi dubbi penso derivino anche della mia non esperienza
riguardo questo metodo.

Un altro punto in cui ho avuto difficolt\`a e che mi distacco da
quello che esprime Beck, riguarda lo stile con cui ho scritto le mie
asserzioni. In \footnote{aggiungere qui riferimento bibliografico a
  TDD, pag. 157}, Beck suggerisce di essere il maggior precisi
possibile nella scrittura di una \emph{assert}, riporto un esempio:
\begin{quotation}
  I've seen assertions like \emph{assertTrue(rectangle.area() !=
    0)}. You could return anything not null and satisfy this test, so
  it isn't very useful. Be specific. If the area should be 50, then
  say that it should be 50: \emph{assertEquals(50, rectangle.area())}.
\end{quotation}
Credendo nei principi espressi nella sezione
\ref{sec:objects-oriented-functional-paradigms}, quindi riducendo
l'uso di metodi accessori, ho avuto difficolt\`a a seguire i consigli
di Beck. Penso per\`o di aver recuperato usando un altra idea che
ritengo molto importante per la scrittura di solidi metodi di test: la
\emph{triangolazione} \footnote{aggiugere qui riferimento
  bibliografico a TDD, pag.153}. 

Questa idea mi ha permesso di utilizzare un approccio molto "chiuso"
per quanto riguarda l'uso di accessori, usando messaggi con lo
stile \emph{isSomethingEquals(...)}, e scrivere lo stesso metodi di
test solidi. Ad esempio:
\begin{lstlisting}
  @Test
  public void matchSpeciesCompartement() {
    Vertex v1 = VertexFactory.makeSimpleVertex();
    Vertex v2 = VertexFactory.makeSimpleVertex();
    Assert.assertNotSame(v1, v2);
    Assert.assertTrue(v1.matchCompartmentWith(v2));
    Assert.assertTrue(v2.matchCompartmentWith(v1));
    Assert.assertFalse(v1.matchSpeciesWith(v2));
    Assert.assertFalse(v2.matchSpeciesWith(v1));
  }
\end{lstlisting}
Usando questo schema, anche se il metodo \emph{matchCompartmentWith()}
invocato sull'oggetto \emph{v1} restituisse un valore fittizio solo
per superare il test, la seguente invocazione su \emph{v2} lo
contraddirrebbe, fallendo il test.
