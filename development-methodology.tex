\section{Metodologia di sviluppo adottata}

La fase di implementazione \`e stata eseguita utilizzando la
metodologia di sviluppo \emph{Test-Driven Development}: abbiamo
cercato di studiare ed applicare quanto Beck espone nel volume
\cite{beck2003}.

Procedere per piccoli passi ed in modo incrementale ci ha permesso lo
sviluppo di un sistema ad oggetti flessibile e mantenibile, avendo
come ulteriore prodotto alla realizzazione della libreria, una
\emph{runnable specification} composta dall'insieme delle batterie di
test implementate.

Il concetto di \emph{Learning Test} ci \`e stato particolarmente utile
nella fase iniziale del progetto, durante la quale abbiamo
familiarizzato con una libreria esterna per il parsing di modelli
codificati con il linguaggio SBML. Nella prossima sezione lo
approfondiremo.

\subsection{Learning tests}
Questa tipologia di test viene introdotta in \cite{beck2003}, pagina
136, ed hanno l'obiettivo di familiarizzare l'utilizzo di una libreria
esterna.

Nel nostro caso abbiamo applicato questi concetti per conoscere la
libreria \emph{JSBML}, la cui documentazione \`e estesa e ben
dettagliata, ma rimangono degli aspetti che non \`e possibile
studiarli a priori.

\`E utile osservare che non solo questi test permettono di apprendere
il funzionamento di un sistema di oggetti fornito da terzi, bens\`i
proteggono il sistema da noi implementato da eventuali problemi
introdotti nei nuovi rilasci della libreria esterna.

Nel nostro caso, se viene fornita una nuova versione della libreria
\emph{JSBML}, abbiamo il grande vantaggio di avere delle asserzioni
che verificano i concetti che sono dipendenze per il nostro lavoro:
siamo interessati che nessuna di queste sia violata da un nuovo
rilascio.

Senza questi test avremmo dovuto provare il software manualmente per
controllare che il comportamento non sia cambiato. Avendoli, invece,
\`e sufficiente esercitare di nuovo tutte le batterie per assicurare
che i concetti su cui ci siamo basati non abbiano subito modifiche.

