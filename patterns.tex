\section{Patterns and coding idioms}

\subsection{Forward specific behaviour from abstract notification}

\subsubsection*{Description}
In questa sezione scrivo una breve descrizione di un interessante
\emph{coding idiom} che ho implementato.
\begin{description}
\item[Problem] Si invoca un metodo il quale ha un parametro con un
  determinato tipo. Questo parametro permette di modellare due
  concetti diversi con lo stesso oggetto. Quello che si vuole \`e di
  poter abilitare un client di ascoltare la computazione ed essere
  notificato usando un oggetto listener sul metodo relativo alla
  specializzazione del concetto che si sta usando attualmente.
\item[Forces] Si vogliono rispettare questi requisiti:
  \begin{itemize}
  \item mantenere una firma del metodo con il parametro di tipo che
    astrae dai diversi concetti. In questo modo il codice non dipende
    dal concetto specifico ma rimane generico.
  \item il client vorrebbe essere notificato non in maniera generica,
    ma sapere quando viene usato un concetto e quando ne viene usato
    uno diverso. L'interfaccia su cui vorrebbe essere notificato il
    listener viene fornita dal client e la possiamo vedere come un
    contratto. 
  \item nel contesto nel quale avviene l'innesco della computazione
    \`e possibile stabilire la natura dei concetti che verranno
    manipolati.
  \end{itemize}
\item[Solution] Per risolvere il problema introduciamo una classe
  astratta che implementa il generico metodo di handling e che
  incapsula il listener fornito dal client, agendo come un
  \emph{wrapper}. Il metodo che esegue la computazione, quando
  incontra un oggetto che modella un concetto, di cui non siamo a
  conoscenza del tipo a \emph{compile time}, invia un messaggio al
  listener. Ma non sar\`a il listener fornito dal client a rispondere
  a tale messaggio, bensi il wrapper. Ma, essendo il wrapper una
  classe astratta, l'oggetto che effettivamente risponder\`a potr\`a
  notificare il vero listener, invocando un messaggio che cattura lo
  specifico concetto che \`e stato elaborato. Lo specifico
  implementatore della classe astratta viene costruito e passato come
  argomento nel contesto in cui \`e possibile sapere il tipo di
  concetti che verranno elaborati dal metodo generico.
\end{description}

\subsubsection*{Reification}
Nella mia implementazione il metodo generico che esegue la
computazione \`e \emph{convertToVertexSet}, il quale ha come parametro
che astrae su due concetti diversi, \emph{listOfSpeciesReference}. 

La computazione eseguita dal metodo \`e di creare un nuovo vertice
ogni qualvolta si trova una species non ancora analizzata, e di
notificare questo vertice al client, tramite il listener passato come
argomento.

\begin{lstlisting}
  public Set<Vertex> convertToVertexSet( 
  ListOf<SpeciesReference> listOfSpeciesReference, 
  Map<Vertex, Vertex> knownVertices,
  VertexHandlingListener listener) 
\end{lstlisting}
Come si vede la firma di questo metodo \`e molto generica, in quanto
il parametro \emph{listOfSpeciesReference} una volta pu\`o contenere
una lista di \emph{reactants}, altre volte una lista di
\emph{products}. Inoltre il listener che viene usato, non mette a
disposizione metodi che catturano il tipo di un concetto specifico,
bensi permette solo di notificare che \`e stato elaborato un vertice,
ma non possiamo sapere se questo \`e un \emph{reactants} o un
\emph{products}.

\`E quindi possibile implementare l'idea data nella parte "Solution",
instanziando il corretto specializzatore della classe astratta nel
contesto dove \`e possibile sapere se si sta per analizzare una
collezione di \emph{reactants} oppure di \emph{products}: questo \`e
possibile farlo nel metodo \emph{readReaction}. Riporto sotto il
momento della creazione del corretto wrapper:
\begin{lstlisting}
Set<Vertex> reactants = this.convertToVertexSet(
				reaction.getListOfReactants(), knownVertices,
				new FromReactantDriver(listener));

Set<Vertex> products = this.convertToVertexSet(
				reaction.getListOfProducts(), knownVertices,
				new FromProductDriver(listener));
                              \end{lstlisting}
Quello che i due wrapper eseguiranno quando verr\`a invocato il metodo
astratto della classe astratta, sar\`a quello di notificare sul
relativo metodo del listener il vertice ricevuto dalla classe
astratta:
\begin{lstlisting}
public abstract class AbstractVertexHandlingListener implements
		VertexHandlingListener {

	private final VertexHandlingWithSourceListener listener;

	protected VertexHandlingWithSourceListener getListener() {
		return listener;
	}

	@Override
	public void vertexHandled(Vertex vertex) {
		this.getListener().vertexHandled(vertex);
		this.onVertexHandled(vertex);
	}

	public abstract void onVertexHandled(Vertex vertex);
}

  protected class FromReactantDriver extends AbstractVertexHandlingListener {

		@Override
		public void onVertexHandled(Vertex vertex) {
			this.getListener().reactantVertexHandled(vertex);
		}
	}

protected class FromProductDriver extends AbstractVertexHandlingListener {

		@Override
		public void onVertexHandled(Vertex vertex) {
			this.getListener().productVertexHandled(vertex);
		}
	}
\end{lstlisting}

\subsection{State pattern}

\subsubsection*{Description}
Riporto la descrizione di questo pattern ripresa da
\cite[p. 327]{SmalltalkCompanion98}:
\begin{quotation}
Allow an object to alter its behaviour when its internal state
changes. The object will appear to change its class.    
\end{quotation}
Nella mia implementazione non ho usato questo pattern ricalcando la
struttura descritta nel riferimento che ho riportato, ma una sua
variante, nella quale durante la vita di un oggetto non si cambia il
suo stato pi\`u volte, bensi viene fissato nel momento della sua
creazione.

\subsubsection*{Reification}
La variante descritta nella sezione precedente mi ha permesso di poter
utilizzare oggetti di tipo \emph{Connector} in due modalit\`a.
\begin{itemize}
\item La prima pi\`u orientata al testing, che permette di inviare
  messaggi ad oggetti \emph{Connector} per leggere delle reazioni e
  verificare degli invarianti sul risultato della lettura. Questa
  modalit\`a non necessita di mantenere nessuna informazione di stato
  all'interno degli oggetti, ma sono di utilizzare parte del loro
  comportamento. La classe interna \emph{StatelessConnectorState}
  implementa quanto detto.
\item la seconda invece \`e quella orientata ad un utilizzo concreto
  nella computazione, la quale permette di effettuare il parsing delle
  reazioni ricevendo in input il \emph{file path} del modello che si
  vuole analizzare. La classe \emph{StatefullConnectorState}
  implementa quanto detto.
\end{itemize}

Questo mi ha permesso di non "sporcare" il codice che implementa le
responsabilit\`a della classe \emph{Connector} con strutture di
controllo per determinare quale delle due modalit\`a si doveva
utilizzare ed inoltre, senza l'applicazione di questo pattern, alcuni
oggetti avrebbero avuto valorizzate delle propriet\`a interne mentre
altri no: avremmo avuto una ridondanza di informazioni e una
cattiva gestione della memoria.
\\\\
Riporto sotto l'estratto di codice nel quale si associa lo stato in
fase di costruzione degli oggetti:
\begin{lstlisting}
  public static Connector makeConnector() {
    Connector connector = new Connector();
    connector.state = connector.new StatelessConnectorState(); 
    return connector; 
  }

  public static Connector makeConnector(String path) { 
    Connector connector = new Connector();
    connector.state = connector.new StatefullConnectorState(path);
    return connector;
  }
\end{lstlisting}
Dato che oggetti di tipo \emph{Connector} sono \emph{value object},
una volta creati non mutano il loro stato, quindi \`e sufficiente
associare il loro stato (forse meglio dire "ruolo") in costruzione.