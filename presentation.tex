\documentclass{beamer}

\usetheme{Boadilla} 
% \setbeamertemplate{blocks}[rounded][shadow=false]
% \useoutertheme{umbcfootline} 
% \setfootline{\insertshortinstitute, \insertshortdate 
%     \hfill slide \insertframenumber/\inserttotalframenumber} 

\title{Analisi di reti metaboliche basata su propriet\`a di
  connessione}
\author{Massimo Nocentini}
\date{Firenze, \today}
%\institute{Universit\`a degli Studi di Firenze}

\begin{document}

\begin{frame}[plain]
  \titlepage
  \begin{center}
    \includegraphics[scale=.065]{logo/unifi}
  \end{center}
\end{frame}

\frame{\tableofcontents}

 
\section*{Introduction}  
\begin{frame}
	\begin{block}{Original source of the following notes}
    Use case driven object modeling with UML - Theory and Practice,\\
    Doug Rosenberg and Matt Stephens,\\
    Apress editor
    \end{block}
\end{frame}
 
\frame
{ 
  \frametitle{Some important points}
  
  \begin{itemize}
  \item capturing your users’ actions and the associated system responses
  \item use case modeling involves analyzing both the \textbf{basic course} (
	typical “sunny-day” usage of the system) and the \textbf{alternate
	courses} (what happens when things go wrong, or when the user tries some
	infrequently used feature of the program)

  \item The use case is written in the context of the domain model that is, all
  	the terms (nouns and noun phrases) that went into the domain model should
  also be used directly in your use case text 
  \item  ICONIX approach assumes that the initial domain model is wrong and
  provides for incrementally improving it as you analyze the use cases
  \item Driving Your Design (and Your Tests) from the Use Cases
  
  \end{itemize}
}

\section{Follow the Two-Paragraph Rule}

\begin{frame}
  \frametitle{Follow the Two-Paragraph Rule}
  \framesubtitle{The proof uses \textit{reductio ad absurdum}.}
  Each use case should fit comfortably into two paragraphs, including both the 
  basic course and alternate courses.
  
  The writer should just write about how the users will be using the system and
 	what the system will do in response.
 	
 	\begin{block}{Helping questions}
 When writing use cases, you need to keep asking the following three
 fundamental questions:
		\begin{itemize} 
		  \item \emph{What happens?} this is the client action gets your
		  “sunny-day scenario” started.
		  \item \emph{And then what happens?} this is the response of the system 
			(keep asking this question until your “sunny-day scenario” is complete.)
		  \item \emph{What else might happen?} (Keep asking this one until you’ve 
		  	identified all the “rainy-day scenarios” you can think of, and
		  	 described the related behavior.)
		\end{itemize}    
     \end{block}
\end{frame}

\section{Organize Your Use Cases with Actors and Use Case Diagrams}
 
\begin{frame}
  \frametitle{Organize Your Use Cases with Actors and Use Case Diagrams}
  %\framesubtitle{The proof uses \textit{reductio ad absurdum}.}
      \begin{itemize}
        \item A use case diagram shows multiple use cases on the one diagram
        \item It’s an overview of a related group of use cases
        \item The text in each oval is the use case title 
      \end{itemize}
 \end{frame}
\begin{frame}
 	\begin{block}{Notations}
		\begin{itemize} 
		  \item \emph{An actor} is represented on the diagram as a stick figure and
		  is analogous to a \textbf{role} that users can play.
		  The actor is \emph{external} to the system, is on the outside area of the
		  diagram, whereas the system is on the inside area. Actors can represent
		  nonhuman external systems as well as people.
		  \item An \emph{association} from the actor to a use case means that the
		  actor is the one who carries out that use case.
		  \item The association can also signify \textbf{responsibilities}. 
		  	For example, an Administrator pointing to a Moderate Forum Messages
			 use case means “The administrator is responsible for moderating forum
			 messages.”
		\end{itemize}    
     \end{block}
\end{frame}

\section{Write Your Use Cases in Active Voice}
 
\begin{frame}
  \frametitle{Write Your Use Cases in Active Voice}
  %\framesubtitle{The proof uses \textit{reductio ad absurdum}.}
      \begin{itemize}
        \item Passive sentences are often unclear and they lack energy
        \item Active sentences make clear who does what
      \end{itemize}
      \begin{block}{Good example}
      The user enters her username and password, and then clicks the Login
      button.\\ 
      The system looks up the user profile using the username and
      checks the password. \\
      The system then logs in the user.
      \end{block}
 \end{frame}

\section{Write Your Use Case Using an Event/Response Flow}
 
\begin{frame}
  \frametitle{Write Your Use Case Using an Event/Response Flow}
A use case is often triggered by a user-initiated event that the system has to 
respond to. However, it can also be triggered by a system-initiated event to
which the user responds.\\ 
      \begin{block}{Refined example}
      The system displays the Login screen. The user enters her username and password, and
then clicks the Login button. The system looks up the user profile using the username
and checks the password. The system then logs in the user.
      \end{block}
 \end{frame}
\begin{frame}
\begin{itemize}
        \item It’s important to remember to write both sides of the user/system dialogue
        \item Use case modeling can be thought of as an outside-in type of approach
        \item A use case will typically consist of several steps.
			Each step involves an event and a response: the user’s action and the
		system’s reaction, or vice versa
        \item You need to write about the user side of the dialogue to keep
        your behavior requirements firmly userfocused
        \item it’s not sufficient to just write down what the user does, because ultimately
you’re trying to spec software, and software really consists of the system’s behavior
      \end{itemize}
\end{frame}

\section{Write Your Use Case in the Context of the Object Model}
 
\begin{frame}
  \frametitle{Write Your Use Case in the Context of the Object Model}
      \begin{alertblock}{Oss}
      You can’t drive object-oriented designs from use cases unless you tie your use cases
to objects
      \end{alertblock}
      This means that you need to reference domain classes that participate in 
      the use case, and you need to name your screens and other boundary objects
explicitly in the use case text

 \end{frame}

\section{Write Your Use Cases Using a Noun-Verb-Noun Sentence Structure}
 
\begin{frame}
  \frametitle{Write Your Use Cases Using a Noun-Verb-Noun Sentence Structure}
                                         
  \begin{itemize}
        \item use case text follows the \textbf{noun-verb-noun} style
       \item The nouns are the object instances. These usually either come from
       the domain model (entities) or are boundary/GUI objects.
        \item The verbs are the messages between objects. These represent the software functions
(controllers) that need to be built.
 \end{itemize}
 \end{frame}

\section{Reference Domain Classes by Name}
  
\begin{frame}
  \frametitle{Reference Domain Classes by Name}
                                         
  \begin{itemize}
        \item the domain model serves as a project glossary                                                                    hat helps to
ensure consistent usage of terms when describing the problem space
        \item it’s critically important that the use cases are linked to the objects
       \item referencing the domain classes by name in the use case text
  \end{itemize}
  \begin{block}{Example using domain object}
   he user selects a Book and adds it to his Wish List. The system displays a page with the
updated list and also displays the user’s Shopping Cart.
  \end{block}
instead of
\begin{alertblock}{Wrong use case}
The user selects a title and adds it to his list of books to be saved for later. The system
displays a page with the updated list and also shows a list of titles in the user’s cart,
ready for checkout.
\end{alertblock}
 \end{frame}
 
\section{Final Example}
 
\begin{frame}
  \frametitle{Final Example}
\begin{block}{Basic Course}
The Customer clicks the Write Review button for the book currently being viewed, and
the system shows the Write Review screen. The Customer types in a Book Review, gives it
a Book Rating out of five stars, and clicks the Send button. The system ensures that the
Book Review isn’t too long or short, and that the Book Rating is within one and five
stars. The system then displays a confirmation screen, and the review is sent to a Moder-
ator, ready to be added.
\end{block} 
\end{frame}
 
\begin{frame}
	\begin{alertblock}{Alternate Course}
		\begin{itemize}
		  \item \textbf{User not logged in} The user is first taken to the 
		 	Login screen and then to the Write Review screen once he is logged in.
		  \item \textbf{The user enters a review that is too long (text $>$ 1MB)}: 
			The system rejects the review and responds with a message explaining why the
			review was rejected.
		\item \textbf{The review is too short ($<$ 10 characters)}: The system
		rejects the review.
		\end{itemize}
	\end{alertblock} 
\end{frame}

\end{document}
 

