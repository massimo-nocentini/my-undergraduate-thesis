\subsection{Pacchetto dotInterface}
\label{subsection:dotInterface-package-description}
Questo pacchetto incapsula tutte quelle astrazioni necessarie alla
creazione di un documento compilabile con i motori messi a
disposizione dalla libreria \emph{graphviz}.

\subsubsection*{Funzionalit\`a implementate}

Le funzionalit\`a fornite da questo pacchetto sono le seguenti:
\begin{itemize}
\item definire i contratti fondamentali che codificano le
  responsabilit\`a affinch\`e un oggetto sia esportabile in formato
  dot e, in modo duale, sia capace di esportare in formato dot un
  altro oggetto. Questi due contratti sono le astrazioni pi\`u
  importanti di tutto il pacchetto e sono dipendenze delle classi
  \emph{Vertex} e \emph{OurModel};
\item incapsulare in un'unica classe tutte quelle funzionalit\`a e
  caratteristiche utilizzate durante la generazione dell'output,
  alcune di esse dipendenti dal sistema operativo che ospita la
  \emph{Java Virtual Machine}, ad esempio: il separatore utilizzato
  nei percorsi di file e directory nel file system, posizioni degli
  output della libreria e delle cartelle dove trovare i modelli di
  input;
\item fornire un wrapper per oggetti di tipo \emph{Writer}, per
  concatenare alla rappresentazione testuale dell'oggetto il carattere
  `;', i due caratteri di \emph{carriege return} e \emph{new line} per
  appendere il prossimo contenuto su una nuova linea, agevolando la
  scrittura di documenti dot.
\end{itemize}

\subsubsection*{Classi}
Procediamo con ordine nel descrivere le principali classi:

\begin{paragraph}{DotExporter}
  Il contratto \emph{DotExporter} definisce quali sono i messaggi che
  un esportatore deve essere in grado di comprendere per costruire
  documenti in formato dot, lavorando su un grafo codificato con le
  astrazioni di \emph{Vertex} e \emph{OurModel}.

  I messaggi definiti in \emph{DotExporter} riguardano le componenti
  basilari di un grafo che si possono rappresentare graficamente: il
  vertice (con eventuale etichetta esterna al nodo) e l'arco (di cui
  non \`e stato necessario introdurre un'etichetta). Riporto il codice
  per maggior chiarezza:
  \begin{lstlisting}
public interface DotExporter extends DotDocumentPartHandler {
  DotExporter buildVertexDefinition(Vertex vertex);
  DotExporter collectCompleteContent(Writer outputPlugObject);
  DotExporter buildEdgeDefinition(Vertex source, Vertex neighbour);
  DotExporter buildVertexLabelOutsideBoxDefinition(Vertex vertex);
}

public interface DotExportable {
  void acceptExporter(DotExporter exporter);
}

  \end{lstlisting}
  Per non accoppiare in modo forte la modalit\`a di salvataggio
  dell'output nei messaggi dell'interfaccia, ne viene esposto uno che
  ha come argomento un oggetto di tipo \emph{Writer}, appartenente alla
  libreria di \emph{io} fornita in \emph{openjdk}.

  Questo non limita l'insieme di destinazioni dell'output, bens\`i
  lascia aperte molte scelte ed eventuali client del contratto
  \emph{DotExporter} potranno utilizzare i propri oggetti come
  destinazione dell'output, potendoli usare per computazioni
  successive. Nelle nostre implementazioni abbiamo usato quasi sempre
  oggetti di tipo \emph{FileWriter} come destinazioni concrete.
\end{paragraph}

\begin{paragraph}{DotExportable}
  Questo contratto permette di definire quali responsabilit\`a devono
  essere implementate affinch\`e un oggetto sia esportabile in formato
  dot.

  Non \`e un'interfaccia molto ricca, ma il solo messaggio
  \emph{acceptExporter} \`e sufficiente per catturare la nostra
  idea. Questa struttura si avvicina molto a quella che viene esposta
  nel pattern \emph{Visitor} in \cite{SmalltalkCompanion98}, anche se
  la nostra implementazione non la ricalca fedelmente: il taglio che
  abbiamo voluto dare alla coppia \emph{DotExporter} e
  \emph{DotExportable} usa lo stesso il \emph{double dispatch}, con la
  differenza di non esporre nel contratto \emph{DotExporter} i metodi
  relativi alle classi concrete di \emph{Vertex}, avendo un solo
  messaggio \emph{buildVertexDefinition} come sopra riportato.
\end{paragraph}




