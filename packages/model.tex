\subsection{Pacchetto Model}
\label{subsection:model-package-description}
Questo pacchetto contiene le implementazioni dei concetti che riguardano
le pi\`u importanti e ricche astrazioni del progetto.

\subsubsection*{Funzionalit\`a implementate}
\label{subsection:model-supplied-abstractions}
Le funzionalit\`a fornite da questo pacchetto sono le seguenti:
\begin{itemize}
\label{itemize:model-supplied-abstraction}
\item definire l'astrazione di vertice e delle sue specializzazioni,
  necessarie per implementare gli altri moduli del
  progetto. Quest'astrazione pu\`o essere modellata a livello teorico
  con un \emph{abstract data type}, in quanto non vogliamo che i
  client siano a conoscenza delle specifiche realizzazioni (come
  invece lo sarebbe se fosse stato un \emph{algebraic data type}),
  vogliamo invece usare un approccio trasparente per mantenere
  flessibile il sistema. Le specializzazioni del concetto di vertice
  che abbiamo implementato sono quelle di vertice semplice, di vertice
  da usare nella visita in profondit\`a e negli algoritmi per la
  ricerca delle componenti fortemente connesse. Per la creazione di
  questi oggetti si utilizzano degli oggetti \emph{factory} (vedi
  relativo pattern in \cite{SmalltalkCompanion98});
\item implementare l'astrazione di grafo, arricchendola con
  comportamenti per poterne modificare la struttura ed eseguire delle
  computazioni su di esso: \`e possibile usare due idee prese dal
  paradigma funzionale per agire sui vertici del grafo ed \`e fornita
  una implementazione della visita in profondit\`a che espone i punti
  di estensione come esposto nella Sezione
  \ref{subsection:dfs-extension-point};
\item interfacciare le astrazioni di vertice e grafo con gli oggetti
  definiti nel pacchetto \emph{DotInterface}, per darne una
  rappresentazione grafica che riporti le idee definite nell'articolo
  \cite{tellingStories}.
\end{itemize}

\subsubsection*{Classi}
Procediamo con ordine nel descrivere le idee principali catturate in
ogni contratto:

\begin{paragraph}{Vertex}
  L'interfaccia \emph{Vertex} \`e l'astrazione principale dell'intera
  libreria.  Tramite \emph{Vertex} possiamo richiedere ad un valido
  implementatore di gestire le informazioni sul vicinato di un
  vertice, dando la possibilit\`a sia di aggiungere vicini che
  predecessori (quindi modelliamo sia una lista di adiacenza, sia una
  lista di incidenza). Questo ci permette di avere un costo
  nell'ordine $O(m + n)$ dove $m$ \`e il numero totali di archi ed $n$
  \`e il numero di nodi presenti nel grafo codificato nel nostro
  modello. Inoltre, per poter collassare le sorgenti in una unica, \`e
  possibile richiedere la ``rottura'' della relazione di vicinato per
  poter successivamente cancellare il vertice dal grafo.

  \emph{Vertex} permette di richiedere informazioni sulle
  caratteristiche del vertice, ad esempio se un vertice \`e sorgente o
  pozzo, qual \`e il suo vicinato, quali sono le informazioni che lo
  distinguono (riferite alla metabolito e al compartimento recuperate
  dal modello SBML di origine).

  Tramite questo contratto possiamo interfacciare le implementazioni
  definite nel pacchetto \emph{dotInterface}, richiedendo di accettare
  un esportatore per costruire un documento in formato \emph{dot}.

  \`E possibile richiedere anche delle informazioni di carattere
  testuale rappresentandole in una tabella, dalle quali si pu\`o
  capire meglio la struttura del grafo.

  Una propriet\`a che questo contratto richiede \`e quella di
  riscrivere i metodi \emph{equals} e \emph{hashCode} in modo che
  tutti gli oggetti implementatori di questa interfaccia possano
  essere trattati come \emph{value objects}, non distinguendoli per
  riferimento in memoria (implementazione di default del metodo
  \emph{equals} fornito da \emph{openjdk}), bens\`i per le
  informazioni che questi incapsulano (e quindi poter usare
  \emph{late-binding} ed avere comportamento polimorfo in base
  all'oggetto che riceve i suddetti messaggi). Abbiamo effettuato
  questa scelta in modo da poter creare implementatori di
  \emph{Vertex} all'occorrenza con le stesse caratteristiche, senza
  dover gestire una struttura dati per la memorizzazione e la ricerca
  dell'unico oggetto creato.

  Come ultima propriet\`a, questo contratto impone una relazione
  d'ordine totale sull'insieme degli implementatori, in modo da
  rendere gli algoritmi deterministici nel momento di selezione dei
  vertici.
\end{paragraph}

\begin{paragraph}{OurModel}
  La classe concreta \emph{OurModel} implementa l'astrazione grafo da
  utilizzare come modello di dominio per le computazioni descritte nel
  Capitolo \ref{chapter:study}.

  La prima responsabilit\`a di \emph{OurModel} \`e costruire il
  modello in diverse modalit\`a: a partire da un insieme di vertici
  gi\`a costruiti (quindi con le relazioni di vicinato gi\`a fissate)
  oppure a partire da un modello SBML esistente in una path del file
  system.

  La seconda \`e templetizzare l'algoritmo della visita in
  profondit\`a, introducendo dei punti di estensione sui quali \`e
  possibile ``personalizzare'' il comportamento della visita ed
  implementare delle varianti. Quelle che abbiamo implementato in
  questo lavoro sono quelle definite nel Capitolo
  \ref{chapter:theoretical-background}, ma questo non limita un
  utilizzatore della libreria di definire il proprio comportamento e
  di usare l'implementazione della visita per algoritmi che non
  abbiamo trattato.

\end{paragraph}

\begin{paragraph}{Implementazioni di Vertex}
  Abbiamo molti comportamenti diversi che vogliamo poter utilizzare
  come istanze dell'astrazione definita dal contratto
  \emph{Vertex}. Il vantaggio di aver implementato il codice
  riferendosi sempre al contratto e non alle singole caratterizzazioni
  \`e di essere trasparente e poter interscambiare i comportamenti
  specifici, lasciando immutato il codice che implementa i vari
  algoritmi.

  Vediamo quali sono gli implementatori del contratto \emph{Vertex},
  descrivendoli brevemente.

  La prima implementazione introdotta \`e stata \emph{SimpleVertex},
  che cattura il comportamento di un vertice ``normale'', ovvero
  l'implementazione pi\`u semplice possibile del contratto. Anche se
  pu\`o sembrare scontata, \`e il mattone utilizzato pi\`u o meno
  indirettamente dalle implementazioni pi\`u complesse: queste sono
  dei \emph{wrapper} che delegano la maggior parte del comportamento a
  questa implementazione base, ridefinendo solo i metodi per i tratti
  che le distinguono.

  La seconda implementazione introdotta \`e stata
  \emph{DfsWrapperVertex}, che associa l'analogo di un ``timestamp''
  nei momenti in cui la visita in profondit\`a raggiunge un vertice e
  in cui lo abbandona. 

  La terza implementazione introdotta \`e stata
  \emph{ConnectedComponentWrapperVertex}, in coppia con
  \emph{TarjanWrapperVertex}. Queste implementano, in modo puramente
  orientato agli oggetti, l'algoritmo per la ricerca delle componenti
  fortemente connesse. In particolare
  \emph{ConnectedComponentWrapperVertex} rappresenta una componente
  fortemente connessa e mantiene l'insieme dei vertici del grafo di
  origine di cui \`e composta, mentre \emph{TarjanWrapperVertex}
  cattura la responsabilit\`a di mantenere la relazione di vicinato
  tra le componenti fortemente connesse. Inoltre abbiamo introdotto
  l'astrazione \emph{ConnectedComponentInfoRecorder} per poter
  sviluppare un semplice programma per la visualizzazione di semplici
  statistiche come descritto nel Capitolo \ref{chapter:study}.

  L'implementazione \emph{VertexWithLabelWrapperVertex}, ortogonale
  alle precedenti, riporta un'etichetta sopra alla rappresentazione di
  un vertice nella generazione dell'output grafico. Questo
  comportamento \`e possibile utilizzarlo in modo trasparente e
  permette di fattorizzare il codice che cattura la decorazione da
  quello che cattura le particolari implementazioni del contratto
  \emph{Vertex}.

  Tutte le precedenti implementazioni sono nascoste ai client del
  contratto \emph{Vertex}, utilizzando \emph{VertexFactory} come unico
  mezzo per poterle costruire, esponendo un metodo statico per ogni
  implementazione che abbiamo descritto.
\end{paragraph}

