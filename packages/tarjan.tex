\subsection{Pacchetto Tarjan}
\label{subsection:tarjan-package-description}
Questo pacchetto non contiene molte astrazioni limitandosi a definire un
contratto fondamentale per implementare la visita \emph{DFS} e
l'algoritmo per la ricerca di componenti fortemente connesse.

\subsubsection*{Funzionalit\`a implementate}
Le funzionalit\`a fornite da questo pacchetto sono le seguenti:
\begin{itemize}
\item definire il contratto che stabilisce gli eventi salienti che
  avvengono durante la visita in profondit\`a del grafo. Questi eventi
  sono codificati come messaggi che possono essere inviati ad un
  implementatore del contratto, notificando lo stato in cui si trova
  la visita. Questo permette di introdurre i punti di estensione come
  descritto nella Sezione \ref{subsection:dfs-extension-point};
\item fornire due implementatori del contratto descritto nel punto
  precedente: uno per costruire un albero \emph{DFS}, l'altro per
  costruire il meta grafo ottenuto ricercando le componenti fortemente
  connesse.
\end{itemize}

\subsubsection*{Classi}
Procediamo con ordine nel descrivere le idee principali catturate
dalle seguenti classi:

\begin{paragraph}{DfsEventsListener}
  Questo contratto definisce gli stati, in cui la visita in
  profondit\`a pu\`o entrare, necessari alle nostre
  implementazioni. Riportiamo sotto la sua definizione direttamente
  dal relativo file sorgente:

\begin{lstlisting}
package tarjan; 
public interface DfsEventsListener {

  void searchCompleted(Map<Vertex, ExploreStatedWrapperVertex> map);
  void postVisit(Vertex v);
  void preVisit(Vertex v);
  void searchStarted(Map<Vertex, ExploreStatedWrapperVertex> map);
  void newVertexExplored(Vertex explorationCauseVertex, Vertex
    vertex);
  void fillCollectedVertices(Set<Vertex> vertices);
  void alreadyKnownVertex(Vertex vertex);
  }
\end{lstlisting}
Il motore che utilizza quest'interfaccia \`e la classe
\emph{OurModel}. Con la definizione di questo contratto possiamo
implementare la visita in profondit\`a in stile orientato agli
oggetti, allontanandosi da una pi\`u comune implementazione
procedurale. Crediamo sia interessante vederla, per cui ci
dilungheremo in questi paragrafi nella sua descrizione. L'inizio della
computazione \`e il seguente metodo definito nella classe
\emph{OurModel}:
\begin{lstlisting}
  public OurModel runDepthFirstSearch(DfsEventsListener
  dfsEventsListener) { 
    final Map<Vertex, ExploreStatedWrapperVertex> map = makeDfsVertexMetadataMap();
    dfsEventsListener.searchStarted(map);
    for (Entry<Vertex, ExploreStatedWrapperVertex> entry : map.entrySet()) {
      entry.getValue().ifNotExplored(dfsEventsListener,	new
      ExploreStateWrapperVertexMapper() 
      {
        @Override
	public ExploreStatedWrapperVertex map(Vertex vertex) {
          return map.get(vertex);
	}
      });
    }
    dfsEventsListener.searchCompleted(map);
    return this;
  }
\end{lstlisting}
Come si nota, mancano alcuni pezzi per una visita corretta,
incapsulati nella classe \emph{ExploreStateWrapperVertex}. Abbiamo
preso questa decisione per avere un sistema pi\`u modulare, rispetto a
codificare tutto in un unico metodo. La responsabilit\`a di
quest'ultima classe \`e di associare ad ogni vertice l'informazione se
questo \`e stato visitato oppure no durante la visita, in modo da
``chiedere'' ai vertici non ancora visitati di esplorare il loro
vicinato, inviando il messaggio \emph{ifNotExplored} che riportiamo:
\begin{lstlisting}
public class ExploredStateWrapperVertex...
  public ExploreStatedWrapperVertex ifNotExplored(
  final DfsEventsListener dfsEventsListener,
  Vertex explorationCauseVertex,
  final ExploreStateWrapperVertexMapper mapper) {
    final Vertex vertex = getWrappedVertex();
    if (isExplored() == false) {
      toggle();
      if (explorationCauseVertex != null) {
        dfsEventsListener.newVertexExplored(explorationCauseVertex,vertex);
      }
      dfsEventsListener.preVisit(vertex);
      vertex.doOnNeighbors(new VertexLogicApplierWithNeighborhoodRelation() {
        @Override
        public void apply(Vertex parent, Vertex neighbour) {
          if ((parent == vertex) == false) {
            throw new RuntimeException("Semantic error");
          }
          mapper.map(neighbour).ifNotExplored(dfsEventsListener,parent, mapper);
        }
      });
      dfsEventsListener.postVisit(vertex);
    } else {
      dfsEventsListener.alreadyKnownVertex(vertex);
    }
    return this;
}
\end{lstlisting}
Questo \`e il blocco mancante nel metodo precedente e che
effettivamente caratterizza la visita in profondit\`a. Nel metodo
sopra riportato si nota che \`e stato semplice implementare questa
parte in quanto, data la natura ricorsiva della struttura, non
dobbiamo far altro che invocare lo stesso metodo su tutti i vertici
del vicinato. Sar\`a in base al loro stato che l'invocazione si
propagher\`a ai rispettivi vicinati. 

Facciamo un'ultima osservazione riguardo agli eventi notificati: i
listener concreti non sono obbligati a definire della logica per ogni
evento, in quanto ognuno di questi viene creato per implementare una
specifica variante e non necessariamente ogni evento \`e richiesto per
raggiungere quanto desiderato.
\end{paragraph}

\begin{paragraph}{DfsEventsListenerTreeBuilder}
  Questo listener associa ad ogni vertice delle informazioni
  necessarie per la costruzione dell'albero \emph{DFS}. In particolare
  si mantiene una coppia di istanti $(t_{in}, t_{out})$ dove $t_{in}$
  rappresenta l'istante in cui il vertice viene raggiunto e $t_{out}$
  l'istante in cui si \`e finito di visitare il rispettivo vicinato.

  Inoltre, nella gestione dell'evento \emph{newVertexExplored}, si
  costruisce la nuova relazione di vicinato, includendo solo quegli
  archi che hanno raggiunto vertici non ancora visitati.

  Riprendendo l'osservazione fatta al termine del paragrafo
  precedente, questo listener non ha bisogno di definire nessuna
  logica per l'evento \emph{alreadyKnownVertex}.
\end{paragraph}

\begin{paragraph}{TarjanEventsListenerTreeBuilder}
  Questo listener implementa l'algoritmo per la ricerca delle
  componenti fortemente connesse, utilizzando la variante descritta
  nella Sezione \ref{subsection:crescenzi-gambosi-grossi}.

  Questo grafo contiene come vertici degli oggetti che hanno
  comportamento specifico relativo alle componenti fortemente connesse
  per cui, nelle successive computazioni, sar\`a possibile usare il
  grafo in modo trasparente, inviando messaggi polimorfi che
  produrranno il comportamento desiderato (ad esempio la l'etichetta
  nella rappresentazione grafica sar\`a diversa rispetto a quella che
  si pu\`o ottenere dopo una visita in profondit\`a).
\end{paragraph}



