\subsection{Pacchetto Piping}
\label{subsection:piping-package-description}
Questo pacchetto contiene le implementazioni che permettono di
assemblare una \emph{pipeline}, attraverso la composizione di un
numero quanto si voglia di filtri. Descriveremo i filtri che abbiamo
implementato e come possiamo crearne di nuovi.

\subsubsection*{Funzionalit\`a implementate}

Le funzionalit\`a fornite da questo pacchetto sono le seguenti:
\begin{itemize}
\item definire un \emph{template} di filtro, componente atomica per la
  composizione di una \emph{pipeline}. In questo template codifica il
  comportamento necessario affinch\`e possa essere combinato,
  lasciando all'implementatore il compito di sviluppare la logica che
  caratterizza il filtro che st\`a modellando;
\item fornire un insieme di filtri gi\`a implementati necessari alla
  realizzazione delle funzionalit\`a descritte nella Sezione
  \ref{section:use-cases};
\item catturare ed esporre il concetto di \emph{listener}, mediante il
  quale \`e possibile gestire degli eventi che avvengono durante la
  computazione. \`E possibile assemblare la propria pipeline ed,
  attraverso un listener, eseguire non solo le trasformazioni che
  questa produce, ma anche della logica aggiuntiva all'accadere di
  determinati eventi. Questo \`e uno strumento molto potente in quanto
  permette di essere ortogonali alla pipeline ed aver maggior
  controllo sull'intera computazione.
\end{itemize}

\subsubsection*{Classi}
Procediamo con ordine nel descrivere le idee principali catturate
dalle seguenti classi:

\begin{paragraph}{PipeFilter}
  Abbiamo cercato di formalizzare in modo preciso la pi\`u piccola
  unit\`a atomica capace di eseguire una computazione all'interno di
  una sequenza pi\`u grande. Questo ci ha portato alla definizione
  della classe astratta \emph{PipeFilter}. Averla introdotta ci ha
  permesso di rendere il codice flessibile e trasparente, nonch\'e
  pi\`u facile da implementare.

  Questa classe, come abbiamo detto nei paragrafi precedenti, in
  realt\`a rappresenta solo un \emph{template} del concetto di filtro
  e non di un filtro che esegue una trasformazione sull'istanza di
  input. Con questo template \`e possibile assemblare una sequenza di
  filtri, usando l'idea della lista concatenata, ed eseguire la
  computazione propagando la richiesta di esecuzione a tutti i filtri
  che si sono assemblati.

  Quello appena detto \`e stato implementato dando a \emph{PipeFilter}
  una struttura ricorsiva: ad esempio per costruire una pipeline con
  due filtri \`e sufficiente costruirli, fissare la relazione di
  precedenza che identifica la pipeline ed invocare la richiesta di
  esecuzione sull'ultimo filtro. Tale filtro, ricevendo la richiesta,
  controlla se il filtro che lo precede esiste: se si allora delega la
  richiesta di esecuzione e si riapplica questo procedimento in modo
  ricorsivo, altrimenti esegue la trasformazione per cui \`e stato
  creato. Inoltre, sempre sfruttando la struttura ricorsiva di
  \emph{PipeFilter}, \`e possibile comporre una pipeline i cui filtri
  sono a loro volta pipeline.

  Ogni filtro lavora su un grafo in input, apporta le modifiche che
  implementa e restituisce in output il grafo trasformato. Questa
  trasformazione da grafi a grafi permette di non avere complessit\`a
  aggiuntiva per quanto riguarda la coerenza dei formati di
  input/output tra filtri adiacenti.
\end{paragraph}

\begin{paragraph}{Implementazioni di PipeFilter}
  Come abbiamo detto, \emph{PipeFilter} rappresenta solo un
  \emph{template}: il comportamento che vogliamo dare ad ogni filtro
  deve essere codificato in una classe concreta che "completa" tale
  template. In realt\`a non \`e molto il lavoro che si lascia da
  scrivere: per completare il template \`e sufficiente implementare un
  metodo astratto, che ha come parametri il grafo di input e deve
  ritornare un grafo utilizzato come input per il filtro successivo.

  Tutti i filtri che abbiamo implementato in questa libreria
  utilizzando il comportamento esposto da altri oggetti.

  Il primo filtro concreto che abbiamo introdotto \`e stato
  \emph{PrinterPipeFilter}, il quale produce una rappresentazione
  grafica compilando un documento \emph{dot}: per adempiere a questo
  compito, costruisce un oggetto di tipo \emph{DotExporter} nel quale,
  durante la visita del grafo, colleziona le informazioni necessarie
  per costruire il documento da compilare. Questo filtro non apporta
  nessuna trasformazione al grafo di input e lo restituisce cos\`i
  come lo ha ricevuto.

  Il secondo filtro concreto che abbiamo introdotto \`e stato
  \emph{DfsPipeFilter}, il quale esegue una visita in profondit\`a del
  grafo in input e ritorna un grafo il cui insieme di archi \`e
  composto da tutti gli archi percorsi dalla visita. Concatenando un
  filtro di tipo \emph{PrinterPipeFilter} sar\`a possibile
  visualizzare la foresta di alberi \emph{DFS} rappresentante la
  visita eseguita.

  Il terzo filtro concreto che abbiamo introdotto \`e stato
  \emph{TarjanPipeFilter}, il quale ricerca le componenti fortemente
  connesse sul grafo di input e restituisce un nuovo \emph{DAG}
  composto dalle componenti identificate. \`E importante notare che
  programmando ``per interfaccie'', il grafo prodotto da questo filtro
  non ha come vertici dei \emph{SimpleVertex}, bens\`i degli oggetti
  che caratterizzano le componenti fortemente connesse. In questo modo
  possiamo continuare ad inviare gli stessi messaggi che possiamo
  inviare ad un oggetto di tipo \emph{SimpleVertex}, ma ottenere una
  comportamento diverso, senza modificare il codice che implementa il
  modello di dominio.

  Il quarto filtro concreto che abbiamo introdotto \`e stato
  \emph{SourcesCollapserPipeFilter}, il quale permette di collassare
  tutte le sorgenti ed introdurne una nuova ed unica. Il nuovo grafo
  viene restituito per eventuali computazioni successive.

  Gli ultimi filtri concreti che abbiamo introdotto sono stati
  \emph{PlainTextStatsPipeFilter} e
  \emph{ConnectedComponentInfoPipeFilter}. Questi hanno la
  responsabilit\`a di rappresentare in formato testuale il grafo in
  input, producendo rispettivamente un output tabellare ed una una
  struttura dati che mantiene delle informazioni sulle componenti
  fortemente connesse, consultabili utilizzando il relativo programma
  di visualizzazione che abbiamo sviluppato.

\end{paragraph}

\begin{paragraph}{PipeFilterComputationListener}
  Il concetto di listener \`e molto simile a quello di \emph{Observer}
  come descritto in \cite{SmalltalkCompanion98}, anche se nella nostra
  implementazione non notifichiamo un cambiamento, bens\`i l'accadere
  di un determinato evento. Modelliamo un evento con dei messaggi
  nell'interfaccia \emph{PipeFilterComputationListener}.

  L'utilit\`a dei listener risiede nella possibilit\`a di ricevere
  degli argomenti inviati dal mittente della notifica per elaborarli
  nel modo desiderato. Questi argomenti possono essere relativi non
  solo ad uno specifico filtro, bens\`i ad un insieme di filtri, in
  quanto si usa lo stesso listener per tutta la durata della
  pipeline. Un esempio \`e \emph{PlainTextInfoComputationListener},
  che colleziona in una mappa avente come chiavi oggetti di tipo
  filtro, e come valori informazioni sulla composizione del
  grafo. Queste informazioni vengono raccolte dapprima durante
  l'esecuzione di un filtro \emph{PlainTextStatsPipeFilter} e, quando
  il processo termina, si invia un messaggio al listener allegando le
  informazioni collezionate.
\end{paragraph}

