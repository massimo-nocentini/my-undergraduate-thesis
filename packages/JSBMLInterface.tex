\subsection{Pacchetto JSBMLInterface}
\label{subsection:JSbmlInterface-package-description}
Questo pacchetto contiene le implementazioni dei concetti che
riguardano l'interfacciamento con modelli SBML utilizzando la libreria
\emph{JSBML} (vedi \cite{JSbmlDistribution}).

\subsubsection*{Funzionalit\`a implementate}

Le funzionalit\`a fornite da questo pacchetto sono le seguenti:
\begin{itemize}
\item astrarre dalla libreria \emph{JSBML} e dal suo modello dati. In
  questo modo l'unico contesto del progetto dipendente dalla libreria
  \emph{JSBML} \`e confinato a questo singolo pacchetto e gli altri
  pacchetti non dovranno essere a conoscenza di come viene codificato
  il modello. Usando questa strategia se si vorr\`a sostituire la
  libreria di interfacciamento con modelli SBML, sar\`a necessario
  apportare le modifiche solo in questo pacchetto, lasciando tutto il
  restante codice del progetto inalterato;
\item interpretare il modello fornito dalla libreria \emph{JSBML},
  costruire gli elementi fondamentali del nostro modello dati e
  renderlo disponibile per successive computazioni.
\end{itemize}

\subsubsection*{Classi}
In questo pacchetto la classe principale \`e \emph{Connector}, la
quale incapsula la responsabilit\`a di delegare alla libreria
\emph{JSBML} la lettura di un modello SBML e, successivamente,
interpretare il risultato della lettura per costruire un nostro
modello dati interno, input di successive computazioni.

La parte di interpretazione del modello letto dalla libreria
\emph{JSBML} \`e quella pi\`u interessante in quanto elabora e
seleziona solo quelle informazioni del modello SBML che effettivamente
sono necessarie al nostro lavoro.

Questo procedimento \`e implementato principalmente nel metodo
\emph{readReactions}, il quale permette di costruire un insieme di
vertici a partire da una collezione di reazioni, i quali verranno
usati per costruire il modello di dominio.
