\documentclass[twoside,openright,titlepage,fleqn,
headinclude,11pt,a4paper,BCOR5mm,footinclude ]{scrbook}
%--------------------------------------------------------------
        \newcommand{\myTitle}{Analisi di reti metaboliche basata su
          propriet\`a di connessione\xspace}
% use the right myDegree option
\newcommand{\myDegree}{Corso di Laurea in Informatica\xspace}
%\newcommand{\myDegree}{
	%Corso di Laurea Specialistica in Scienze e Tecnologie 
	%dell'Informazione\xspace}
\newcommand{\myName}{Massimo Nocentini\xspace}
\newcommand{\myProf}{Pierluigi Crescenzi\xspace}
\newcommand{\myOtherProf}{Nome Cognome\xspace}
\newcommand{\mySupervisor}{Nome Cognome\xspace}
\newcommand{\myFaculty}{
	Facolt\`a di Scienze Matematiche, Fisiche e Naturali\xspace}
\newcommand{\myDepartment}{
	Dipartimento di Sistemi e Informatica\xspace}
\newcommand{\myUni}{\protect{
	Universit\`a degli Studi di Firenze}\xspace}
\newcommand{\myLocation}{Firenze\xspace}
\newcommand{\myTime}{Anno Accademico 2010-2011\xspace}
\newcommand{\myVersion}{Version 0.1\xspace}
%--------------------------------------------------------------
\usepackage[latin1]{inputenc} 
\usepackage[T1]{fontenc} 
\usepackage[square,numbers]{natbib} 
\usepackage[fleqn]{amsmath}  
\usepackage[italian]{babel}
%--------------------------------------------------------------
\usepackage{dia-classicthesis-ldpkg} 
%--------------------------------------------------------------
% Options for classicthesis.sty:
% tocaligned eulerchapternumbers drafting linedheaders 
% listsseparated subfig nochapters beramono eulermath parts 
% minionpro pdfspacing
\usepackage[eulerchapternumbers,subfig,beramono,eulermath,
	parts]{classicthesis}
%--------------------------------------------------------------
\newlength{\abcd} % for ab..z string length calculation
% how all the floats will be aligned
\newcommand{\myfloatalign}{\centering} 
\setlength{\extrarowheight}{3pt} % increase table row height
\captionsetup{format=hang,font=small}
%--------------------------------------------------------------
% Layout setting
%--------------------------------------------------------------
\usepackage{geometry}
\geometry{
	a4paper,
	ignoremp,
	bindingoffset = 1cm, 
	textwidth     = 13.5cm,
	textheight    = 21.5cm,
	lmargin       = 3.5cm, % left margin
	tmargin       = 4cm    % top margin 
}
%--------------------------------------------------------------
\usepackage{listings}
\usepackage{hyperref}
% My Theorem
\newtheorem{oss}{Observation}[section]
\newtheorem{exercise}{Exercise}[section]
\newtheorem{thm}{Theorem}[section]
\newtheorem{cor}[thm]{Corollary}

\newtheorem{lem}[thm]{Lemma}

\newcommand{\vect}[1]{\boldsymbol{#1}}

% questo comando e' relativo alle correzioni che puo
% apportare il prof se lo desidera.
\newcommand{\prof}[1]{\boldsymbol{#1}}

% instead of boldsymbol I can use the arrow above the letter with
%\newcommand{\vect}[1]{\vec{#1}}

% page settings
% \pagestyle{headings}
%--------------------------------------------------------------
\begin{document}
\frenchspacing
\raggedbottom
\pagenumbering{roman}
\pagestyle{plain}
%--------------------------------------------------------------
% Frontmatter
%--------------------------------------------------------------
%\include{titlePage}
\pagestyle{scrheadings}
%--------------------------------------------------------------
% Mainmatter
%--------------------------------------------------------------
\pagenumbering{arabic}

% settings for the lstlisting environment
\lstset{
	language = java
	, numbers = left 
	, basicstyle=\sffamily%\footnotesize
	%, frame=single
	, tabsize=2
	, captionpos=b
	, breaklines=true
	, showspaces=false
	, showstringspaces=false
}

\chapter*{Analisi di reti metaboliche basata su propriet\`a di
  connessione}
\begin{description}
\item[Candidato] Massimo Nocentini, \url{massimo.nocentini@gmail.com}
\item[Relatore] Pierluigi Crescenzi, \url{pierluigi.crescenzi@unifi.it}
\end{description}
Il nostro elaborato tratta l'analisi di reti metaboliche e si collega
ad una variante del problema di enumerare tutti i sotto grafi aciclici
di un grafo orientato, aggiungendo il vincolo che solo un determinato
sottoinsieme $\mathbb{B}$ di vertici possono avere il ruolo di
sorgente o di pozzo nei \emph{DAG} enumerati. Ogni \emph{DAG} che
soddisfa tale condizione \`e chiamato \emph{storia} e si vuole
``raccontare'' tutte le possibili storie dato un grafo orientato e
l'insieme $\mathbb{B}$. Modellando una rete metabolica con la
struttura astratta grafo \`e possibile ricercare le storie in essa
contenute: il nostro lavoro verifica se \`e possibile costruire
l'insieme $\mathbb{B}$ in modo automatico per condurre tale ricerca.

Per raggiungere il nostro obiettivo abbiamo dapprima costruito il
grafo da studiare a partire dalla rete metabolica codificata con il
linguaggio \emph{SBML} e, successivamente, ne abbiamo analizzato la
connettivit\`a, ricercando le sue componenti fortemente connesse. Una
volta costruito il ``meta grafo'' che ha come vertici le componenti
fortemente connesse e come archi gli archi che collegano metaboliti
contenuti in componenti diverse (evitando duplicazioni), assegniamo un
ruolo ad ogni componente $c$: se $c$ \`e \emph{sorgente} o
\emph{pozzo} allora possiamo inserire i vertici che la compongono
nell'insieme $\mathbb{B}$.

Per automatizzare questo processo abbiamo prodotto una libreria,
sviluppata con il linguaggio Java, nella quale vengono implementati
gli algoritmi per la visita in profondit\`a e per la ricerca delle
componenti fortemente connesse. Questa libreria non \`e orientata ad
essere utilizzata come un programma a se stante, bens\`i mira ad un
utilizzo programmatico ed estendibile. Nella libreria \`e presente
anche una maschera realizzata utilizzando il framework \emph{SWING}
per la renderizzazione di un particolare insieme di risultati,
relativi alla composizione delle componenti fortemente connesse, utile
per indagare la distribuzione dei ruoli ai metaboliti considerando un
insieme di reti metaboliche.

Abbiamo applicato le nostre implementazioni a due insiemi di reti
metaboliche, analizzando circa 170 modelli contenenti circa 12500
metaboliti, osservando che se ci limitiamo ad indagare una singola
rete metabolica allora il nostro processo \`e in grado di costruire
l'insieme $\mathbb{B}$, mentre se consideriamo un insieme di reti
metaboliche la costruzione di $\mathbb{B}$ \`e affetta da errori in
quanto esistono metaboliti che ``interpretano'' un ruolo diverso in
modelli diversi.



\end{document}
