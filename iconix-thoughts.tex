\section{ICONIX method difficult to see here}
Durante il corso di \emph{Tecniche di programmazione} ho avuto
opportunit\`a di sviluppare un progetto utilizzando una metodologia
sviluppata da Doug Rosemberg e Matt Stephens, chiamata \emph{ICONIX},
la quale mira ad improntare lo sviluppo partendo da dei documenti di
specifica del comportamento e, in modo iterativo, raffinarli (in
seguito a successive convalidazioni) in modo da produrre dei documenti
tecnici di pi\`u basso livello, contenenti tutte le informazioni
richieste per una implementazione chiara e precisa \footnote{Su questa
  tecnica ho avuto modo di discutere e avere delle correzioni da due
  colleghi, Arash Tavallay e Emanuele Conti, che ringrazio molto per i
  loro consigli}.

Riporto un passo preso dal volume che introduce
\emph{ICONIX}\footnote{aggiungere qui riferimento bibliografico a USe
  Case Driven, pag. 50} che ritengo catturi le idee chiavi da usare
come punto di partenza:
\begin{quotation}
  \textbf{Don't Forget the Rainy-Day}: Scenarios when you're writing
  your use cases, write them in such a way that your efforts are
  focused on capturing your users' actions and the associated system
  responses. As you'll see, use case modeling involves analyzing both
  the basic course (a user's typical "sunny-day" usage of the system;
  often thought of as 90\% of the behavior) and the alternate courses
  (the other 10\% of the system functionality, consisting of
  "rainy-day" scenarios of the way in which the user interacts with
  the system; in other words, what happens when things go wrong, or
  when the user tries some infrequently used feature of the
  program). If you capture all of this in your use cases, you have the
  vast majority of the system specced out.
\end{quotation}

Nonostante ci siano degli aspetti che non condivido appieno del metodo
(in particolare sui \emph{domain model}, come il lettore si render\`a
conto al termine delle prime sezioni del capitolo
\ref{chapter:implementation}), ci sono comunque dei punti di forza (in
particolare i \emph{robustness diagrams}) che avrei avuto piacere
utilizzare durante questo lavoro, di cui invece non ho potuto giovarne
dei vantaggi.

La maggior difficolt\`a che ho trovato nello stilare \emph{use-case}
seguendo le linee di \emph{ICONIX} sono dovute al fatto che il mio
lavoro non ha una natura interattiva (anche se si potrebbe obiettare
che l'attore \emph{user} sia a sua volta un programma e non un utente
umano) e applicare la regola \emph{two-paragraph rule}
\footnote{aggiungere riferimento bibliografico al volume Use Case
  Driven, pag. 52} nella descrizione degli use-case non portava
vantaggi nella specifica delle mie idee.

Per un esempio di quello che si potrebbe ottenere utilizzando tale
regola, non applicato ad \emph{toy examples}, bensi ad un progetto
reale, si rimanda a \footnote{aggiungere qui riferimento bibliografico
  alla tesi di Arash e Ema}.

Per i motivi sopra esposti, nelle prossime sezioni le descrizione
degli use-case non sar\`a molto formale ma avr\`a una forma pi\`u
discorsiva.