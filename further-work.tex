
\section{TODO}
\begin{itemize}
\item dire che l'azione di compattare tutte le sorgenti \`e nata dal
  risultato della ricerca delle componenti fortemente connesse in
  quanto avevamo troppe sorgenti.
\end{itemize}

\section{Further work}

In questa sezione elenco alcuni punti che possono essere presi come
basi di partenza per sviluppi futuri del progetto. 

\begin{description}
\item[dependency injection] Rifattorizzare le parti del codice in cui
  vengono utilizzati degli oggetti \emph{factory} per nascondere la
  costruzione di realizzazioni specifiche di interfaccie (ad esempio
  come descritto in \nameref{itemize:model-supplied-abstraction}),
  sostituendole con motori di \emph{dependency injection}, applicando
  il principio di \emph{invertion of control}.
\item[domain specific language] speficare e implementare un \emph{DSL}
  per poter descrivere e assemblare la pipeline in modo dichiarativo,
  senza dover scendere al livello delle interfaccie e delle classi che
  abbiamo implementato. Questa dichiarazione potrebbe essere scritta o
  direttamente nella riga di comando oppure in un file esterno.
\item[graphviz interface] utilizzare una libreria che permette di
  usare programmaticamente gli oggetti e le implementazioni fornite
  dalla libreria \emph{graphviz}. Nell'attuale lavoro si utilizzano
  tali oggetti solo eseguendoli in modalit\`a di comando, equivalente
  ad una invocazione dal terminale \emph{bash}. Questo permetterebbe
  di avere un maggior grado di portabilit\`a del lavoro svolto,
  eliminando la dipendenza da una installazione a monte di
  \emph{graphviz}.
\item[configuration file] creare un file di configurazione nel quale
  si potrebbero impostare quale motore di \emph{graphviz} utilizzare
  per la renderizzazione dei grafi (ad esempio esistono anche
  \emph{neato} e molti altri) e il preambolo dei documenti dot nei
  quali si specifica le formattazioni, i colori di nodi e archi e le
  rispettive dimesioni.

  Inoltre potrebbe essere interessante esprire quanto detto nel
  precedente paragrafo con uno stesso DSL, in modo da poter esprire le
  proprie pipeline in batterie e lasciare al codice il codice di
  costruire i necessari oggetti ed eseguire le rispettive
  computazioni.
  % put here another further-work
\end{description}
