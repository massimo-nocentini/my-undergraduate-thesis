
\section{TODO}
\begin{itemize}
\item dire che l'azione di compattare tutte le sorgenti \`e nata dal
  risultato della ricerca delle componenti fortemente connesse in
  quanto avevamo troppe sorgenti.
\end{itemize}

\section{Further work}

In questa sezione elenco alcuni punti che possono essere presi come
basi di partenza per sviluppi futuri del progetto. 

\begin{description}
\item[dependency injection] Rifattorizzare le parti del codice in cui
  vengono utilizzati degli oggetti \emph{factory} per nascondere la
  costruzione di realizzazioni specifiche di interfaccie (ad esempio
  come descritto in \nameref{itemize:model-supplied-abstraction}),
  sostituendole con motori di \emph{dependency injection}, applicando
  il principio di \emph{invertion of control}.
\item[domain specific language] speficare e implementare un \emph{DSL}
  per poter descrivere e assemblare la pipeline in modo dichiarativo,
  senza dover scendere al livello delle interfaccie e delle classi che
  abbiamo implementato. Questa dichiarazione potrebbe essere scritta o
  direttamente nella riga di comando oppure in un file esterno.
\end{description}
