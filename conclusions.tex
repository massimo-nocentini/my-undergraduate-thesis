\section{Risultati}
In questa sezione riportiamo le informazioni riguardanti le esperienze
che abbiamo effettuato: elencheremo le sorgenti dei modelli di
interesse, come utilizzare la libreria sviluppata per riprodurre le
prove e, in conclusione, analizzeremo il nostro processo di
costruzione dell'insieme $\mathbb{B}$.

\subsection{Modelli biologici studiati}
Da un punto di vista biologico, la maggior parte dei modelli oggetto
dei nostri studi sono riferiti a micro organismi batterici e sono
reperibili in \cite{SymBioCyc} e in \cite{MetExplore} gratuitamente:
quelli che si trovano nel primo riferimento sono stati i primi su cui
abbiamo lavorato, mentre i secondi sono quelli su cui abbiamo
esercitato in modo massivo le nostre implementazioni.

Se si trovassero difficolt\`a nella loro ricerca, le due sorgenti sono
disponibili in \cite{MyJavaImpl}, descritto nella Sezione
\ref{section:used-tools}. Dopo aver scaricato il repository, i modelli
relativi a \cite{SymBioCyc} si trovano all'interno della cartella
\emph{sbml-test-files}, mentre quelli relativi a \cite{MetExplore}
nell'archivio \emph{sbml-test-files/BioCyc15.0.tar.gz}.

\subsection{Come riprodurre le nostre esperienze}
Ogni prova che abbiamo effettuato \`e codificata in un \emph{test
  method}, indipendente dalle altre, per cui \`e sufficiente compilare
i sorgenti reperibili in \cite{MyJavaImpl}, sia delle classi che dei
metodi di test, ed esercitare con il \emph{TestRunner} che si
preferisce tutta la batteria di test. I risultati vengono salvati
nella cartella \emph{dot-test-files/tests-output}, all'interno
troviamo:
\begin{itemize}
\item i documenti dot interpretabili con \emph{graphviz} e le relative
  rappresentazioni sotto forma di immagine \emph{svg}, risultato di
  quanto descritto nelle Sezioni
  \ref{subsection:represent-it-in-black-and-white},
  \ref{subsection:apply-depth-first-search} e
  \ref{subsection:use-case-tarjan};
\item i file testuali contenenti informazioni sulla struttura dei
  grafi riferiti sia a dei modelli ``ad hoc'' riportati anche in
  questo documento, sia a reti metaboliche, delle quali non \`e stato
  possibile riportare una rappresentazione data la loro grande
  dimensione. Questi file sono il risultato di quanto descritto nelle
  Sezioni \ref{subsection:tabular-representation-use-case} e
  \ref{subsection:collapse-sources-use-case};
\item le strutture dati contenenti la composizione delle componenti
  fortemente connesse, relative ai modelli presenti nelle sorgenti
  citate nella sezione precedente. Per visualizzarle, usiamo la
  maschera descritta nella Sezione
  \ref{subsection:use-case-result-viewer}, eseguendo la classe
  util.ResultViewer.
\end{itemize}

\subsection{Osservazioni su quanto ottenuto}
Le osservazioni che possiamo fare sui risultati delle prove possono
essere divise in base agli obiettivi che ci poniamo:
\begin{itemize}
\item se si vuole costruire l'insieme $\mathbb{B}$ usando una sola
  rete metabolica allora \`e possibile assegnare un ruolo ad ogni
  metabolito: per visualizzare l'assegnazione \`e sufficiente
  selezionare nella \emph{list-box} alla destra della tabella il ruolo
  e, in risposta, il sistema selezioner\`a, nella rispettiva
  \emph{list-box}, ogni metabolito a cui \`e associato il ruolo
  selezionato. Questo \`e il caso proposto nella Figura (AGGIUNGERE
  QUI LA FIGURA RELATIVA);
\item se si vuole costruire l'insieme $\mathbb{B}$ usando un insieme
  di reti metaboliche allora si puo' avere incoerenza e perdita di
  precisione. Considerando i modelli in \cite{MetExplore} vediamo che
  esistono dei metaboliti che non hanno un singolo ruolo e questo,
  seppur con delle percentuali basse, pu\`o introdurre errori nella
  costruzione dell'insieme $\mathbb{B}$. Questo \`e il caso proposto
  nella Figura (AGGIUNGERE QUI LA FIGURA RELATIVA);
% \item facendo riferimento all'articolo
%   \cite{large-scale-reconstruction} di cui abbiamo accennato nella
%   Sezione \ref{section:existing-works}, se si vuole identificare un
%   insieme di \emph{compound}, usando il termine originale, tra cui
%   scegliere i rappresentanti dei \emph{seed sets} \`e sufficiente
%   selezionare nella \emph{list-box} alla destra della tabella la
%   tipologia \emph{Sources} e scegliere un metabolito tra quelli
%   selezionati in automatico nella rispettiva \emph{list-box}. Questo
%   \`e il caso proposto nella Figura ();
\item se si vuole semplificare la rete utilizzando il meta grafo delle
  componenti fortemente connesse osserviamo che, per tutti i modelli
  studiati, non abbiamo un grafo significativo in quanto vi sono molte
  componenti sorgenti e pozzo, mentre sono molto poche le componenti
  intermedie. Questo \`e il caso proposto nella Figura ().
\end{itemize}
