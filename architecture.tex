\section{L'architettura \emph{pipes and filters}}

In questa sezione descriveremo l'architettura che vogliamo dare al
nostro lavoro, coscienti degli obiettivi espressi nelle precedenti
sezioni e delle loro caratteristiche.

Daremo prima alcune idee prese dagli articoli che hanno introdotto
quest'architettura e, nella seconda parte della sezione, vedremo come
possiamo ``metterle a punto'' nel nostro progetto.

\subsection{Cenni teorici}
Quest'architettura \`e molto usata nella pratica e vi \`e molta
letteratura da cui poter attingere, pensiamo per\`o che tornare alle
sorgenti che per prime hanno permesso la sua diffusione sia di
notevole importanza prima di analizzare le varianti e gli studi
successivi.

I volumi da cui ho studiato sono quello del Buschmann \cite{POSA} e
dalla miscellanea di Coplien e Schmidt \cite{PLOPD}.

\subsubsection{Schema classico}
Nella testata che descrive il pattern in \cite{POSA}, pagina 53,
compare:
\begin{quotation}
  The \emph{Pipes and Filters} architectural pattern provides a
  structure for systems that process a stream of data. Each processing
  step is encapsulated in a filter component. Data is passed through
  pipes between adjacent filters. Recombining filters allows you to
  build families or related systems.
\end{quotation}
L'idea alla base di questo pattern \`e il principio \emph{separation
  of concerns}. Quello che si vuol fare \`e dividere tutte le
responsabilit\`a del sistema in sotto problemi, ognuno dei quali pu\`o
essere implementato con una componente software tale che \emph{non
  abbia dipendenze} comportamentali dalle altre (non \`e client di
nessun'altra, non vi \`e delega di comportamento) e che \emph{coopera}
usando l'output di una componente come input per un'altra.

\`E interessante quello che aggiungono Coplien e Schmidt in
\cite{PLOPD}, pagina 431:
\begin{quotation}
  Usually, transformation of input data is done locally and
  incrementally, so that output may begin before the input is
  completely read. This means that a filter may start to work as soon
  as its predecessor produces its first result.
\end{quotation}
Quest'idea \`e molto importante se vogliamo introdurre un grado
maggiore di parallelismo nel sistema oppure distribuire la
computazione. Nel nostro lavoro questo non \`e stato possibile farlo,
in quanto le computazioni che abbiamo incapsulato nei nostri filtri
non permettono di produrre un output che possa ritenersi un
semilavorato, pronto per essere raffinato dal filtro successivo.

\subsubsection{Conseguenze sull'uso dei filtri}
Riportiamo ancora un'osservazione fatta in \cite{PLOPD}, pagina 431:
\begin{quotation}
  To preserve the independence of the framework's components, filters
  may not share a state. The only way to put the results of different
  filters together is to organize some of them into a sequence such
  that certain filters perform further transformations on the outputs
  of others. In addition, a filter should not know the identity of the
  filters preceding or following it in the computation sequence.
\end{quotation}
La funzionalit\`a \ref{subsection:use-case-tarjan} \`e in un certo
senso quello che Coplien identifica nella frase ``...\emph{certain
  filters perform further transformations...}'' per la funzionalit\`a
\ref{subsection:use-case-result-viewer}, in quanto quest'ultima assume
di lavorare su un grafo minimizzato utilizzando un algoritmo per la
ricerca delle componenti fortemente connesse.

Aver introdotto formalmente il concetto di filtro come astrazione \`e
stato molto vantaggioso in quanto ogni computazione, sia effettiva che
di "adattamento", pu\`o essere trattata indipendentemente e in modo
trasparente alla stregua delle altre.

\subsubsection{Conseguenze sull'uso delle pipes}
Riporto ancora da Coplien, pagina 432:
\begin{quotation}
  Two principally different possibilities exist for the realization of
  pipes: pipes may simply be links between filters (such as message
  calls) or they may be separate components (such as data repositories
  or sensors). A pipe's only responsibility is to transmit data
  between filters, eventually by converting their format from the one
  produced by their sender to the one required by their receiver.
\end{quotation}
Nel nostro lavoro non abbiamo avuto bisogno della complessit\`a di
modellare veri e propri oggetti \emph{pipe} in quanto, come vedremo
nella prossima sezione, non \`e necessario adattare il formato
dell'input per coppie diverse di filtri, responsabilit\`a di oggetti
\emph{pipe}. Abbiamo quindi scelto la modalit\`a ``\emph{links between
  filters}''.

\subsection{Implementazione in questo progetto}
L'implementazione dell'architettura ``\emph{pipes and filters}'' in
questo progetto ha cercato di prendere spunto da quanto detto nella
sezione precedente ma allo stesso tempo ha delle caratteristiche che
la distinguono dalla linea introdotta.

Abbiamo introdotto la classe \emph{PipeFilter} per modellare il
concetto di \emph{filtro} e, come sua propriet\`a, un riferimento al
filtro successivo, per modellare il concetto di \emph{pipe}.

Inoltre possiamo vedere la \emph{pipeline} come una coda di filtri,
dove il primo filtro ad essere inserito nella pipeline \`e il primo ad
essere eseguito. Utilizzando la terminologia usata da Buschmann, la
nostra implementazione ricade nello scenario ``\emph{pull pipeline}'',
dove la computazione viene innescata dalla richiesta di un client.

Non vi \`e bisogno di convertire l'output di un filtro per esser
trattato come input per un altro, in quanto ogni filtro lavora con
l'astrazione \emph{OurModel}, anche se avremo comportamento diverso in
base al tipo di oggetti \emph{Vertex} contenuti all'interno
dell'input\footnote{aggiungere qui riferimento all'appendice
  riguardante i precedenti concetti}.

La precedente osservazione potrebbe indurre il lettore a pensare alla
nostra architettura non tanto come una pipeline, quanto come un
sistema di decoratori (vedi pattern \emph{Decorator} in
\cite{SmalltalkCompanion98}). Quest'interpretazione non \`e errata,
anzi crediamo che indebolire il requisito di stabilire un ordinamento
rigido dei filtri presente nel pattern \emph{pipeline} originale, a
favore di maggior dinamicit\`a e trasparenza sul formato di
input/output, porti ad una soluzione che sarebbe stato sufficiente
implementare con un tipico schema con decoratori.
