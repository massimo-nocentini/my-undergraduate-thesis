\section{Pure object-oriented and functional programming paradigms}
\label{sec:objects-oriented-functional-paradigms}

In questa sezione descriveremo le idee fondamentali e i principi che
abbiamo cercato di rispettare durante la fase di
implementazione.

Questi concetti vengono ripetuti ed usati molte volte nel codice: in
alcuni casi possono rendere la lettura del codice e lo studio pi\`u
problematico rispetto alle controparti "procedurali", ma hanno i
grandi valori aggiunti di portare maggior flessibilit\`a e
manutenibilit\`a delle astrazioni.

Ogni sottosezione ha come titolo una massima, che pu\`o sembrare un po
"stravagante" ma penso che sia un potente mezzo di comunicazione.

\subsection{Getters and setters are evil}
Riporto quello che dice Holub nel suo volume \footnote{aggiungere qui
  riferimento bibliografico a "Holub on Pattern", pag. 27}:
\begin{quotation}
  [...] it's a fundamental precept of OO systems that an object not
  expose any of its implementation details. This way, you can change
  the implementation without needing to change the code that uses the
  object. It follows that you should avoid getter and setter
  functions, which typically do nothing but provide access to
  implementation details (fields), in OO systems.
\end{quotation}

Quello che dice Holub ritengo sia il vero principio che permette ai
sistemi di essere \emph{flessibili}, nonostante non sia facile da
rispettare in tutte le situazioni. La programmazione procedurale aveva
come conseguenza l'esistenza di tante funzioni su un insieme di
argomenti, ognuna delle quali si pu\`o considerare un "piccolo
controller", capace di decidere e applicare della logica in base agli
argomenti a disposizione.

Con l'introduzione dell'astrazione \emph{object} e il mantra
\emph{everything is an object}, possiamo ridurre l'esistenza di questi
controller e dare potere decisionale sul comportamento da eseguire
agli oggetti stessi. Di conseguenza gli oggetti non sono delle
\emph{struct}ure passive, ma raggiungono maggior potenza quando quello
che modellano \`e solo comportamento.

Quindi seguendo quello che propone Holub, le informazioni devono
circolare all'interno del sistema, certo, ma se minimizziamo la
presenza di metodi accessori (termine che accomune sia metodi
\emph{get} che \emph{set}) possiamo modellare un sistema dove esistono
solo oggetti che si scambiano messaggi e non dove esistono dei
\emph{chunk} adibiti a immagazzinare informazioni, per essere usati da
oggetti "privilegiati".

Nel mio codice un esempio di quanto affermato sopra \`e il contratto
\emph{Vertex}, il quale, come analizzato nei capitoli precedenti, ha
come informazioni salienti: identificatore, nome, compartimento e
vicinato. Quello che il contratto richiede ad un oggetto per essere
considerato un \emph{Vertex} (detto cos\`i sembra molto lo stile
\emph{duck typing} dei linguaggi dinamici, in Java \`e pi\`u corretto
dire che la classe deve implentare tutte le caratteristiche richieste
dal contratto affinche sulle sue istanze si possano invocare tali
metodi) non \`e \emph{getIdentifier()} o \emph{getNeighborhood()},
bensi:
\begin{lstlisting}
  boolean isYourNeighborhoodEquals(Set<Vertex> products);
  boolean isYourNeighborhoodEmpty();
  boolean isYourSpeciesId(String speciesId);
  boolean isYourCompartmentId(String compartmentId);
  boolean isYourSpeciesName(String species_name);
\end{lstlisting}
quello che si st\`a chiedendo non \`e tanto una azione passiva, ma un
controllo attivo a cui solo il destinatario pu\`o rispondere in modo
univoco.

Si pu\`o andare oltre, in quanto il tipo di ritorno \emph{boolean},
pu\`o invogliare a scrivere logica condizionale (altro \emph{smell}
che ritengo sia da minimizzare), nella prossima sezione vedremo
come si potrebbe far meglio.

\subsection{Don't ask, tell!}

Riporto ancora da Holub, marcato in grassetto nel suo volume
\footnote{sempre riferimento a Holub on Patterns, pag 30}:
\begin{quotation}
  Don't ask for the information that you need to do some work; ask the
  object that has the information to do the work for you.
\end{quotation}
Penso che linguaggi come Smalltalk, Ruby, Lisp incapsulino questo
principio e il linguaggio stesso sia pi\`u ricco per trattare
\emph{everything an object}. Tornando alla promessa fatta alla fine
della sezione precedente, supponiamo di voler eseguire una determinata
logica se il vicinato di un vertice \`e vuoto. In Java dovremo
scrivere:
\begin{lstlisting}
  if(vertex.isYourNeighborhoodEmpty()){
    // do some ugly logic...
  }
\end{lstlisting}
In Smalltalk invece:
\begin{lstlisting}
  vertex isYourNeighborhoodEmpty ifTrue: anUglyBlock
\end{lstlisting}
La differenza \`e che nella versione Java la conoscenza sul vicinato
adesso \`e in due punti, sia nella classe concreta di cui
\emph{vertex} \`e una istanza, sia nella classe a cui appartiene il
blocco di codice riportato.

Nella versione Smalltalk invece la conoscenza sul vicinato \`e
incapsulata all'interno dell'oggetto \emph{vertex} e in nessun altro
punto, in quanto il codice client riportato, passa al metodo
\emph{ifTrue} un \emph{block} da eseguire a discrezione dell'oggetto
\emph{vertex} (\emph{don't ask, tell!}) se il suo vicinato \`e vuoto.

Quanto detto nell'ultima frase porta naturalmente ad un paradigma
chiave del paradigma funzionale, che descriviamo nella prossima
sezione.

\subsection{Functions (or better, behaviours) as values}

