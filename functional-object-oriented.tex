\section{Pure object-oriented and functional paradigms}
\label{sec:objects-oriented-functional-paradigms}

In questa sezione descriveremo le idee fondamentali e i principi che
abbiamo cercato di rispettare durante la fase di implementazione.

Questi concetti vengono ripetuti ed usati molte volte nel codice: in
alcuni casi possono rendere la lettura del codice e lo studio pi\`u
problematico rispetto alle controparti "procedurali", ma hanno i
grandi valori aggiunti di portare maggior flessibilit\`a e
manutenibilit\`a delle astrazioni.

Ogni sottosezione ha come titolo una massima, che pu\`o sembrare un po
"stravagante" ma penso che sia un potente mezzo di comunicazione.

\subsection{Getters and setters are evil}
Riporto quello che dice Holub nel suo volume \footnote{aggiungere qui
  riferimento bibliografico a "Holub on Pattern", pag. 27}:
\begin{quotation}
  [...] it's a fundamental precept of OO systems that an object not
  expose any of its implementation details. This way, you can change
  the implementation without needing to change the code that uses the
  object. It follows that you should avoid getter and setter
  functions, which typically do nothing but provide access to
  implementation details (fields), in OO systems.
\end{quotation}

Quello che dice Holub ritengo sia il vero principio che permette ai
sistemi di essere \emph{flessibili}, nonostante non sia facile da
rispettare in tutte le situazioni. La programmazione procedurale aveva
come conseguenza l'esistenza di tante funzioni su un insieme di
argomenti, ognuna delle quali si pu\`o considerare un "piccolo
controller", capace di decidere e applicare della logica in base agli
argomenti a disposizione.

Con l'introduzione dell'astrazione \emph{object} e il mantra
\emph{everything is an object}, possiamo ridurre l'esistenza di questi
controller e dare potere decisionale sul comportamento da eseguire
agli oggetti stessi. Di conseguenza gli oggetti non sono delle
\emph{struct}ure passive, ma raggiungono maggior potenza quando quello
che modellano \`e solo comportamento.

Quindi seguendo quello che propone Holub, le informazioni devono
circolare all'interno del sistema, certo, ma se minimizziamo la
presenza di metodi accessori (termine che accomune sia metodi
\emph{get} che \emph{set}) possiamo modellare un sistema dove esistono
solo oggetti che si scambiano messaggi e non dove esistono dei
\emph{chunk} adibiti a immagazzinare informazioni, per essere usati da
oggetti "privilegiati".

Nel mio codice un esempio di quanto affermato sopra \`e il contratto
\emph{Vertex}, il quale, come analizzato nei capitoli precedenti, ha
come informazioni salienti: identificatore, nome, compartimento e
vicinato. Quello che il contratto richiede ad un oggetto per essere
considerato un \emph{Vertex} (detto cos\`i sembra molto lo stile
\emph{duck typing} dei linguaggi dinamici, in Java \`e pi\`u corretto
dire che la classe deve implentare tutte le caratteristiche richieste
dal contratto affinche sulle sue istanze si possano invocare tali
metodi) non \`e \emph{getIdentifier()} o \emph{getNeighborhood()},
bensi:
\begin{lstlisting}
  boolean isYourNeighborhoodEquals(Set<Vertex> products);
  boolean isYourNeighborhoodEmpty();
  boolean isYourSpeciesId(String speciesId);
  boolean isYourCompartmentId(String compartmentId);
  boolean isYourSpeciesName(String species_name);
\end{lstlisting}
quello che si st\`a chiedendo non \`e tanto una azione passiva, ma un
controllo attivo a cui solo il destinatario pu\`o rispondere in modo
univoco.

Si pu\`o andare oltre, in quanto il tipo di ritorno \emph{boolean},
pu\`o invogliare a scrivere logica condizionale (altro \emph{smell}
che ritengo sia da minimizzare), nella prossima sezione vedremo
come si potrebbe far meglio.

\subsection{Don't ask, tell!}

Riporto ancora da Holub, marcato in grassetto nel suo volume
\footnote{sempre riferimento a Holub on Patterns, pag 30}:
\begin{quotation}
  Don't ask for the information that you need to do some work; ask the
  object that has the information to do the work for you.
\end{quotation}
Penso che linguaggi come Smalltalk, Ruby, Lisp incapsulino questo
principio e il linguaggio stesso sia pi\`u ricco per trattare
\emph{everything an object}. Tornando alla promessa fatta alla fine
della sezione precedente, supponiamo di voler eseguire una determinata
logica se il vicinato di un vertice \`e vuoto. In Java dovremo
scrivere:
\begin{lstlisting}
  if(vertex.isYourNeighborhoodEmpty()){
    // do some ugly logic...
  }
\end{lstlisting}
In Smalltalk invece:
\begin{lstlisting}
  vertex isYourNeighborhoodEmpty ifTrue: anUglyBlock
\end{lstlisting}
La differenza \`e che nella versione Java la conoscenza sul vicinato
adesso \`e in due punti, sia nella classe concreta di cui
\emph{vertex} \`e una istanza, sia nella classe a cui appartiene il
blocco di codice riportato.

Nella versione Smalltalk invece la conoscenza sul vicinato \`e
incapsulata all'interno dell'oggetto \emph{vertex} e in nessun altro
punto, in quanto il codice client riportato, passa al metodo
\emph{ifTrue} un \emph{block} da eseguire a discrezione dell'oggetto
\emph{vertex} (\emph{don't ask, tell!}) se il suo vicinato \`e vuoto.

Quanto detto nell'ultima frase porta naturalmente ad un paradigma
chiave del paradigma funzionale, che descriviamo nella prossima
sezione.

\subsection{Functions (or better, behaviours) as values}
In \footnote{aggiungere qui riferimento bibliografico a The Reasoned
  Schemer, pag ix}, dopo poche parole della prefazione, compare:
\begin{quotation}
  Our main assumption is that you understand the first eight chapters
  of \emph{The little schemer}. The only true requirement, however, is
  that you understand functions as values. That is, a function can be
  both an argument to and the value of a function call. Furthermore,
  you should know that functions remember the context in which they
  were created.
\end{quotation}
Quello che dice Friedman \`e una caratteristica dei linguaggi
funzionali (quelli usati nel suo volume sono \emph{Scheme} e
\emph{Lisp}) e in molti linguaggi orientati agli oggetti si ritrovano
delle imitazioni di questo concetto.

In \emph{Smalltalk} e \emph{Ruby} esistono i \emph{block}, che
permettono di costruire un oggetto che incapsula un comportamento ma
che verr\`a esiguito in un momento successivo e, dato che esso stesso
\`e un oggetto (\emph{as value}), verr\`a eseguito solo quando
ricever\`a il messaggio \emph{value}.

In \emph{C\#} sono stati introdotti i \emph{delegates} come costrutto
del linguaggio, che permette sia di poter referenziare un metodo
(l'idea assomiglia ai \emph{function pointer} presenti in \emph{C++}
anche se vi sono nette differenze) sia di poter definire un metodo
anonimo da eseguire successivamente, e non al momento della sua
definizione. Con l'evoluzione del linguaggio sono state introdotte le
\emph{lambda-expressions}, che aggiungono dello zucchero sintattico al
linguaggio e rendono la costruzione di metodi anonimi (quello che in
\emph{Lisp} esiste dalla sua creazione nel 1956) pi\`u semplice, anche
se non aggiungono potere espressivo (quello che \`e possibile scrivere
con \emph{lambda-expression} ha un suo equivalente utilizzando il
costrutto \emph{delegate} e viceversa).

Veniamo al linguaggio utilizzato per l'implementazione di questo
lavoro. \emph{Java} non ha costrutti del linguaggio che permettono di
esprimere i concetti espressi sopra, e una valutazione superficiale,
porterebbe a darne un giudizio inferiore rispetto gli altri
linguaggi. Questo non \`e vero secondo me, in quanto pur mancando un
supporto diretto del linguaggio, vengono messi a disposizione i
\emph{building block} essenziali per poter esprimere le stesse idee:
le \emph{interfacce} e la loro implementazione anonima. Tramite questi
due strumenti \`e possibile arrivare allo stesso obiettivo, suppure
con maggior fatica, rispetto ai linguaggi sopra citati. Questo pu\`o
tedioso per\`o, secondo la mia opinione, \`e uno strumento che porta a
capire davvero cosa sia una \emph{lambda function} e quali idee ci
sono dietro.

Ad esempio nel mio codice, quando ho avuto bisogno di eseguire una
determinata logica su ogni vertice nel vicinato di uno dato, come
prima cosa non abbiamo azzardato neppure ad inviare un messaggio
\emph{getNeighbors()} (\emph{getters and setters are evil,
  remember?}), bensi abbiamo lasciato la resposabilit\`a all'oggetto
vertice di applicare una logica su ogni suo vicino, cosa che solo
questo oggetto \`e a conoscenza di come fare. Il problema che rimane
da sciogliere \`e come incapsulare questa logica. Beh, adesso dovrebbe
essere facile, in quanto la nostra logica lavora su oggetti di tipo
\emph{Vertex}, quindi creiamo una interfaccia con un messaggio che lo
ha come parametro:
\begin{lstlisting}
  public interface VertexLogicApplier {
    void apply(Vertex vertex);
  }
\end{lstlisting}
Ogni classe che implementa questa interfaccia \`e una logica che pu\`o
essere usata come argomento al vertice di cui ci interessa il
vicinato. Questo \`e catturato proprio dal contratto \emph{Vertex}:
\begin{lstlisting}
  public interface Vertex...
  void doOnNeighbors(VertexLogicApplier applier);
\end{lstlisting}
Questo ci permette di usare il potente concetto di \emph{functions as
  values} utilizzando due interfaccie. Inoltre, sembrer\`a
contraddittorio, ma abbiamo molta pi\`u flessibilit\`a: quando
definiamo una \emph{lambda function} quello che possiamo fare \`e
codificare un solo comportamento, in quanto questa viene eseguita
quando riceve \emph{un} messaggio (nel caso di \emph{Smalltalk} il
messaggio \`e \emph{value}). In \emph{Java} invece stiamo usando
interfaccie e questo non pone vincoli sul numero di comportamenti che
\`e possibile aggiungere, \`e sufficiente definire nell'interfaccia
tanti messaggi quanti sono necessari.
