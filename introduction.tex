
Il mio lavoro inizia con lo studio e l'utilizzo del formato
\textbf{SBML} (\textbf{S}ystems \textbf{B}iology \textbf{M}arkup 
\textbf{L}anguage). 

Questo formato permette di rappresentare informazioni seguendo uno
schema gerarchico, basato su XML. Il suo obiettivo \`e quello di
facilitare la codifica di modelli computazionali di processi
biologici, pertanto \`e orientato alla descrizione di sistemi in cui
entit\`a biologiche sono oggetto di manipolazioni eseguite da processi
nel corso del tempo.

Con SBML si possono codificare diverse classi di fenomeni biologici:
metabolic networks, cell-signaling pathways, regulatory networks e
molti altri. In particolare, le reti metaboliche sono oggetto del mio
elaborato.
\\\\
Nelle prossime sezioni descrivo brevemente il contesto scientifico che
ospita il nostro problema, introducendo i concetti reali oggetto delle
astrazioni che ho costruito e implementato nel mio codice. Inoltre,
per migliore la comprensione e giustificare l'utilizzo del formato
SBML, propongo un relazione che accoppia i concetti reali con la loro
codifica in un modello SBML.

\section{Metabolic networks, enzimes and pathways}

Prima di definire cosa \`e una metabolic network dobbiamo introdurre i
seguenti concetti. 

Una \emph{metabolic pathway} \`e una sequenza di reazioni chimiche che
si verificano all'interno di una cellula. Dal punto di vista
matematico, possiamo vedere una metabolic pathway come una funzione
$reaction$ tale che:
\begin{displaymath}
reaction : 2^{Molecules} \rightarrow 2^{Molecules}
\end{displaymath}
avendo in input un insieme di molecole, esegue delle reazioni chimiche su
queste e produce come output il risultato delle trasformazioni svolte
sottoforma di insieme di molecole.
\\\\
Ogni reazione chimica \`e regolata da alcuni \emph{enzymes}. 

Un \emph{enzyme} \`e una proteina che gestisce la frequenza e la
velocit\`a di una reazione chimica. Le molecole a cui si applica la
reazione vengono identificate con il termine \emph{substrates} (o
\emph{reactants} usando la terminologia SBML), mentre i prodotti della
reazione vengono indentificati con il termine \emph{products} (anche la
terminologia SBML usa questo identificatore).

Durante l'esecuzione di una reazione chimica, ogni \emph{enzyme}
agisce da \emph{catalyst}, ovvero non viene consumato nella reazione
e, quindi, pu\`o partecipare in pi\`u di una trasformazione.

L'insieme di \emph{enzymes} "guida" e determina l'insieme di
\emph{pathways} che possono occorrere nella cellula, in quanto una
reazione chimica su un substrato pu\`o avvenire se e solo se lo strato
attivo del substrato complementa quello dell'enzima.

Adesso possiamo definire una \emph{metabolic network} come
collezione di \emph{metabolic pathways}.

\section{SBML mapping}

Adesso che abbiamo alcuni concetti reali possiamo iniziare a mapparli
sulle nostre astrazioni, la prima delle quali \`e il mezzo di
comunicazione SBML.

Nella precedente sezione abbiamo introdotto alcuni concetti che non
sono influenti sul nostro studio, per cui trattiamo solo quelli
inerenti al lavoro che ho sviluppato.

I concetti essenziali che interessano questo elaborato sono quelli 
\emph{chemical reaction}, di cui \emph{reactants, products} sono le
entit\`a atomiche che utilizzeremo, e l'attributo \emph{reversible}
per indicare se una reazione \`e reversibile oppure no.

Con questi concetti siamo in grado di costruire, dato un modello
codificato in SBML, la nostra astrazione del concetto di
\emph{metabolic network}.

SBML ha molti altri elementi per catturare quanto spiegato nella
precedente sezione (come gli \emph{enzymes} e vari attributi specifici
di ogni reazione) ma questi non sono necessari al nostro studio.